%!TEX root = ../hauptdatei.tex
\chapter{Anforderungsanalyse}

\section{Begriffsdefinitionen}
Im Folgenden werden die Begriffe, die für die funktionalen und nichtfunktionalen Anforderungen notwendig sind, definiert.

\subsection{Daten} 
Bei dem Begriff \enquote{Daten} handelt es sich um Informationen zu Backprodukten oder Neuigkeiten.

\subsection{System}
Mit dem Begriff \enquote{System} ist die Verwaltung für die hinterlegten Daten gemeint.

\subsection{Administrator}
Mit dem Begriff \enquote{Administrator} ist die Person, die für das Pflegen der Daten zulässig ist, gemeint.

\subsection{Benutzer}
Mit dem Begriff  \enquote{Benutzer} ist eine Technologie zur Anzeige der angelegten Daten gemeint.

\section{Funktionale Anforderungen}

\subsection{[FA10] Zentraler Speicherort} \label{fa10}
Das System muss die Daten zentral in einer Datenbank speichern.

\subsection{[FA20] Zentrale Administration} \label{fa20}
Das System muss dem Administrator über eine zentrale Schnittstelle die Möglichkeit bieten, die Daten pflegen zu können.

\subsection{[FA30] Alle Backprodukte anzeigen} \label{fa30}
Das System muss dem Benutzer und dem Administrator die Möglichkeit bieten, die Daten zu allen Backprodukten, sofern bereits welche angelegt wurden, anzeigen lassen zu können.

\subsection{[FA40] Backprodukt anzeigen} \label{fa40}
Das System muss dem Benutzer und dem Administrator die Möglichkeit bieten, die Daten zu einem ausgewählten Backprodukt, sofern dieses bereits angelegt wurde, anzeigen lassen zu können.

\subsection{[FA50] Backprodukt erstellen} \label{fa50}
Das System muss dem Administrator die Möglichkeit bieten, ein neues Backprodukt anlegen zu können.

\subsection{[FA60] Backprodukt bearbeiten} \label{fa60}
Das System muss dem Administrator die Möglichkeit bieten, ein bestehendes Backprodukt bearbeiten zu können.

\subsection{[FA70] Backprodukt löschen} \label{fa70}
Das System muss dem Administrator die Möglichkeit bieten, ein bestehendes Backprodukt löschen zu können.

\subsection{[FA80] Alle Neuigkeiten anzeigen} \label{fa80}
Das System muss dem Benutzer und dem Administrator die Möglichkeit bieten, die Daten zu allen Neuigkeiten, sofern bereits welche angelegt wurden, anzeigen lassen zu können.

\subsection{[FA90] Neuigkeit anzeigen} \label{fa90}
Das System muss dem Benutzer und dem Administrator die Möglichkeit bieten, die Daten zu einer ausgewählten Neuigkeit, sofern diese bereits angelegt wurde, anzeigen lassen zu können.

\subsection{[FA100] Neuigkeit erstellen} \label{fa100}
Das System muss dem Administrator die Möglichkeit bieten, eine neue Neuigkeit anlegen zu können.

\subsection{[FA110] Neuigkeit bearbeiten} \label{fa110}
Das System muss dem Administrator die Möglichkeit bieten, eine bestehende Neuigkeit bearbeiten zu können.

\subsection{[FA120] Neuigkeit löschen} \label{fa120}
Das System muss dem Administrator die Möglichkeit bieten, eine bestehende Neuigkeit löschen zu können.

\newpage

\section{Nichtfunktionale Anforderungen}

\subsection{[NFA10] Authentifizierung zur Administration} \label{nfa10}
Das System muss die Administration von Daten bei einem nicht autorisierten Zugriff verweigern, sofern es sich nicht um den Administrator handelt.

\subsection{[NFA20] Authentifizierung zur Anzeige von Daten} \label{nfa20}
Das System muss die Anzeige von Daten bei einem nicht autorisierten Zugriff verweigern, sofern es sich nicht um den Benutzer handelt.

\subsection{[NFA30] iOS App mit iPad Kompatibilität} \label{nfa30}
Das System muss eine iOS App mit iPad Kompatibilität bereitstellen, um die Daten für Endnutzer anzeigen lassen zu können.